%!TEX TS-program = xelatex

\documentclass[t]{beamer}

\usetheme{Hannover}
\usecolortheme{rose}

\usepackage{fontspec,xltxtra,xunicode}      %% подготавливает загрузку шрифтов Open Type, True Type и др.
%\defaultfontfeatures{Ligatures={TeX},Renderer=Basic}  %% свойства шрифтов по умолчанию
\setmainfont{Brill} 
\setsansfont{Brill}
\setmonofont[Ligatures=NoCommon]{DejaVu Sans}
\usepackage{amsmath,amsfonts,amssymb,amsthm,mathtools} % AMS
\usepackage{icomma} % "Умная" запятая: $0,2$ --- число, $0, 2$ --- перечисление


%%% Работа с таблицами
\usepackage{array,tabularx,tabulary,booktabs} % Дополнительная работа с таблицами
\usepackage{longtable}  % Длинные таблицы
\usepackage{multirow} % Слияние строк в таблице

%%% Страница
%\usepackage{fancyhdr} % Колонтитулы
% 	\pagestyle{fancy}
 	%\renewcommand{\headrulewidth}{0pt}  % Толщина линейки, отчеркивающей верхний колонтитул
% 	\lfoot{Нижний левый}
% 	\rfoot{Нижний правый}
% 	\rhead{Верхний правый}
% 	\chead{Верхний в центре}
% 	\lhead{Верхний левый}
%	\cfoot{Нижний в центре} % По умолчанию здесь номер страницы

\usepackage{setspace} % Интерлиньяж
%\onehalfspacing % Интерлиньяж 1.5
%\doublespacing % Интерлиньяж 2
\singlespacing % Интерлиньяж 1

\usepackage{hyperref}
\hypersetup{
     colorlinks   = true,
     citecolor = black,
     urlcolor    = blue
}


%%% Лингвистические пакеты
%\usepackage{savetrees} % пакет, который экономит место
\usepackage{natbib}
\bibpunct[: ]{[}{]}{;}{a}{}{,}
%\usepackage{glossary-mcols} 
%\setglossarystyle{mcolindex}
\newcommand{\mytem}{\item[$\circ$]}
\newcommand{\apostrophe}{\XeTeXglyph\XeTeXcharglyph"0027\relax}\setbeamercolor{alerted text}{fg=blue}
\setbeamersize{text margin left=4mm,text margin right=1mm} 
\setbeamertemplate{frametitle}[default][center]
\setbeamertemplate{navigation symbols}{
	\usebeamerfont{footline}%
    \usebeamercolor[fg]{footline}%
    \hspace{1em}%
    {{\small презентация доступна: \href{https://goo.gl/HUrRuk}{\textbf{https://goo.gl/HUrRuk}}}
    \hspace{4cm}
    \insertframenumber/\inserttotalframenumber\vspace{0.5mm}}}
\title[]{Vowels}
\author[]{G. Moroz}
\date{10 February, 2018}
\begin{document}
\frame{\titlepage}

\begin{frame}{Previously}
\begin{itemize}
\item Sound waves have
\begin{itemize}
\item A --- amplitude
\item f --- fundamental frequency
\item φ --- phase
\item t --- time
\end{itemize}
\item Speech sounds are complex waves
\item Fourier transform --- allows to extract components of the complex wave
\end{itemize}
\end{frame}


\begin{frame}{Source-Filter Model}
\begin{itemize}
\item \href{https://raw.githubusercontent.com/agricolamz/2018_m_Instrumental_Phonetics/master/docs/materials/larynx.mp4}{Larynx produce some sound}
\item \href{https://raw.githubusercontent.com/agricolamz/2018_m_Instrumental_Phonetics/master/docs/materials/MRI-speech.mp4}{Vocal tract filter some frequencies}
\end{itemize}
\includegraphics[width=0.95\linewidth]{01-source-filter.png}
\end{frame}

\section{Vowel production}
\begin{frame}{How shape of the vocal tract influences on vowels?}
\Large
\vfill
\begin{center}
\begin{vowel}
\putcvowel[l]{i}{1}
\putcvowel[r]{y}{1}
\putcvowel[l]{e}{2}
\putcvowel[r]{ø}{2}
\putcvowel[l]{ɛ}{3}
\putcvowel[r]{œ}{3}
\putcvowel[l]{a}{4}
\putcvowel[r]{ɶ}{4}
\putcvowel[l]{ɑ}{5}
\putcvowel[r]{ɒ}{5}
\putcvowel[l]{ʌ}{6}
\putcvowel[r]{ɔ}{6}
\putcvowel[l]{ɤ}{7}
\putcvowel[r]{o}{7}
\putcvowel[r]{u}{8}
\putcvowel[l]{ɯ}{8}
\putcvowel[r]{ɨ}{9}
\putcvowel[l]{ʉ}{9}
\putcvowel[r]{ɘ}{10}
\putcvowel[l]{ɵ}{10}
\putcvowel{ə}{11}
\putcvowel[r]{ɜ}{12}
\putcvowel[l]{ɞ}{12}
\putcvowel{ɪ ʏ}{13}
\putcvowel{ʊ}{14}
\putcvowel{ɐ}{15}
\putcvowel{æ}{16}
\end{vowel}
\end{center}
\vfill
\normalsize
Historically, height and backness are impressionistic linguistic terms
\end{frame}

\begin{frame}{How shape of the vocal tract influences on vowels?}
\Large
\vfill
\begin{center}
\begin{vowel}
\putcvowel{i}{1}
\putcvowel{a}{4}
\putcvowel{u}{8}
\end{vowel}
\end{center}
\vfill
\end{frame}

\begin{frame}{How shape of the vocal tract influences on vowels?}
\Large
\vfill
\begin{center}
\begin{vowel}
\putcvowel{\includegraphics[width=0.2\linewidth]{02-i.png}}{1}
\putcvowel{\includegraphics[width=0.2\linewidth]{03-a.png}}{4}
\putcvowel{\includegraphics[width=0.2\linewidth]{04-u.png}}{8}
\end{vowel}
\end{center}
\vfill
\end{frame}

\begin{frame}{How shape of the vocal tract influences on vowels?}
\Large
\vfill
\begin{center}
\begin{vowel}
\putcvowel{\includegraphics[width=0.3\linewidth]{05-i-spectrum.png}}{1}
\putcvowel{\includegraphics[width=0.3\linewidth]{06-a-spectrum.png}}{4}
\putcvowel{\includegraphics[width=0.3\linewidth]{07-u-spectrum.png}}{8}
\end{vowel}
\end{center}
\vfill
\normalsize
\begin{center}
\begin{tabular}{|c|c|c|c|}
\hline
 & i & a & u \\ \hline
F1 & 300 & 700 & 300 \\ \hline
F2 & 2300 & 1400 & 800 \\ \hline
\end{tabular}
\end{center}
\end{frame}

\begin{frame}{Vowel chart}
\includegraphics[width=\linewidth]{03-vowel-chart.png}
\end{frame}

\section{Tube model}
\begin{frame}{How shape of the vocal tract influences on vowels?}
\Large
\vfill
\begin{center}
\begin{vowel}
\putcvowel{\includegraphics[width=0.2\linewidth]{08-i-tube.png}}{1}
\putcvowel{\includegraphics[width=0.2\linewidth]{09-a-tube.png}}{4}
\putcvowel{\includegraphics[width=0.2\linewidth]{10-u-tube.png}}{8}
\end{vowel}
\end{center}
\vfill
\normalsize
Tube model, \citep{fant60}: vocal tract is a tube or a set of tubes
\end{frame}

\begin{frame}{Wavelength}
\includegraphics[width=0.8\linewidth]{11-wavelength.png}
$$c = \frac{\lambda}{T} = \lambda\times f \approx 33400\text{ cm/s}$$
\textit{c} --- speed of sound; \textit{λ} --- wavelength; \textit{f} --- sound frequency; \textit{T} --- period
\end{frame}

\begin{frame}{Neutral vocal tract in the position for the vowel ə}
\includegraphics[width=0.8\linewidth]{12-shwa-tube.png}\\
Resonance is a phenomenon in which a vibrating system or external force drives another system to oscillate with greater amplitude at specific frequencies. The lowest natural frequency at which such a tube resonates will have a wavelength (\textit{λ}) \textbf{four times the length} of the tube (\textit{L}).
$$f = \frac{c}{\lambda} = \frac{c}{4 \times L} \approx \frac{33400}{17.5 \times 4} \approx 477 \text{ Hz} \approx 500 \text{ Hz}$$

The tube also resonates at \textbf{odd multiples} of that frequency. \pause \textbf{Why?}
\end{frame}

\begin{frame}{Wave addition}
\includegraphics[width=\linewidth]{13-waves.jpeg}\\
\end{frame}

\begin{frame}{Neutral vocal tract in the position for the vowel ə}
\includegraphics[width=0.8\linewidth]{12-shwa-tube.png}\\
$$F_1 = \frac{c}{\lambda} = \frac{c}{4 \times L} \approx 500 \text{ Hz}$$
$$F_2 = \frac{c}{\lambda} = \frac{c}{\frac{4}{3} \times L} = \frac{3 \times c}{4 L} \approx 1500 \text{ Hz}$$
$$F_3 = \frac{c}{\lambda} = \frac{c}{\frac{4}{5} \times L} = \frac{5 \times c}{4 L} \approx 2500 \text{ Hz}$$
$$F_n = \frac{c}{\lambda} = \frac{c}{\frac{4}{n} \times L} = \frac{n \times c}{4 L} \approx n \times 500 \text{ Hz}$$
\end{frame}

\begin{frame}{\invisible<2>{???}\visible<2>{Cat meow}}
\includegraphics[width=\linewidth]{14-cat-meow.png}\\
\href{https://github.com/agricolamz/2018_m_Instrumental_Phonetics/raw/master/docs/materials/cat_meow.wav}{listen}
\end{frame}

\begin{frame}{How shape of the vocal tract influences on vowels?}
\Large
\vfill
\begin{center}
\begin{vowel}
\putcvowel{i}{1}
\putcvowel{a}{4}
\putcvowel{u}{8}
\end{vowel}
\end{center}
\vfill
\end{frame}

\begin{frame}{How shape of the vocal tract influences on vowels?}
\Large
\vfill
\begin{center}
\begin{vowel}
\putcvowel{\includegraphics[width=0.2\linewidth]{02-i.png}}{1}
\putcvowel{\includegraphics[width=0.2\linewidth]{03-a.png}}{4}
\putcvowel{\includegraphics[width=0.2\linewidth]{04-u.png}}{8}
\end{vowel}
\end{center}
\vfill
\end{frame}

\begin{frame}{How shape of the vocal tract influences on vowels?}
\Large
\vfill
\begin{center}
\begin{vowel}
\putcvowel{\includegraphics[width=0.2\linewidth]{08-i-tube.png}}{1}
\putcvowel{\includegraphics[width=0.2\linewidth]{09-a-tube.png}}{4}
\putcvowel{\includegraphics[width=0.2\linewidth]{10-u-tube.png}}{8}
\end{vowel}
\end{center}
\vfill
\normalsize
When there is a constriction, back tube and constriction form \href{https://github.com/agricolamz/2018_m_Instrumental_Phonetics/raw/master/docs/materials/Helmholtz-resonance.mp4}{\textbf{Helmholtz resonator}}
$$f = \frac{c}{2\pi} \times \sqrt{\frac{A}{V\times L}}$$
\textit{A} --- the area of the neck; \textit{L} --- length of the tube; \textit{V} --- volume of the air in the body
\end{frame}

\begin{frame}{How shape of the vocal tract influences on vowels?}
\includegraphics[width=0.6\linewidth]{15-tubes.png}
\end{frame}

\section{other models}
\begin{frame}{Other models}
\begin{itemize}
\item Perturbation Theory Kajiyama 1941, Mrayati et al. 1988
\item Quantal Theory Stevens 1989
\item Theory of adaptive dispersion Lindblom 1990
\item \dots
\end{itemize}
\end{frame}

\section{seminar}
\begin{frame}{How to get something like that?}
\includegraphics[width=\linewidth]{03-vowel-chart.png}
\end{frame}

\begin{frame}{Vowel editor}
Praat objects > New > Sound > Create sound from VowelEditor...
\includegraphics[width=\linewidth]{04-vowel-editor.png}
\end{frame}


\section{Vowel analysis}
\begin{frame}{How to analyse vowels?}
\begin{itemize}
\item record sounds
\item annotate sounds
\item make an exploratory analysis
\item extract duration and formant information from your data
\item create the plot
\end{itemize}
\end{frame}

\begin{frame}{How to analyse vowels?}
\begin{itemize}
\item[\checkmark] record sounds
\item[\checkmark] annotate sounds
\item make an exploratory analysis
\item extract duration and formant information from your data
\item create the plot
\end{itemize}
\end{frame}

\subsection{exploratory analysis}
\begin{frame}{Formants in Praat}
Praat Analyser > Formant > Show Formants\\
\includegraphics[width=\linewidth]{05-formnats.png}\\
F1 \hfill select the nearest first formant value or mean value for selection\\
F2 \hfill select the nearest second formant value or mean value for selection\\
F3 \hfill select the nearest third formant value or mean value for selection\\
\end{frame}

\begin{frame}{Formants in Praat}
Praat Analyser > Formant > Formant Settings...\\
\includegraphics[width=\linewidth]{05-formnats.png}\\
During analysis you should set Maximum Formant value  so as to distinguish [i], [a] and [u].
\end{frame}

\begin{frame}{How to analyse vowels?}
\begin{itemize}
\item[\checkmark] record sounds
\item[\checkmark] annotate sounds
\item[\checkmark] make an exploratory analysis
\item extract duration and formant information from your data
\item create the plot
\end{itemize}
\end{frame}

\subsection{extract data}
\begin{frame}{Change writing preferences to UTF-8!}
Praat Objects > Preferences > Text writing preferences...
\end{frame}

\begin{frame}{Praat scripting}
Praat have its own scripting language. You can read about it:\\
Praat Objects > Help > Scripting tutorial\\
There are  a lot of Praat scripts \href{http://phonetics.linguistics.ucla.edu/facilities/acoustic/praat.html}{here}.
\end{frame}

\begin{frame}{Praat scripting: extracting duration}
\begin{itemize}
\item Open Praat Objects
\item Open some TextGrid
\item Praat Objects > Praat > New Praat script
\item Copy script from \href{https://goo.gl/n7vbDy}{here} to the new window
\item Select TextGrid
\item Praat Script > Run > Run
\item Provide some valid path for the result file
\item Press OK
\end{itemize}
\end{frame}

\begin{frame}{Praat scripting: extracting formant values}
\begin{itemize}
\item Open Praat Objects
\item Praat Objects > Praat > New Praat script
\item Copy script from \href{https://goo.gl/MUBpPw}{here} to the new window
\item Praat Script > Run > Run
\item Provide some path with your sound and TextGrid
\item Provide Maximum Formant value
\item Press OK
\end{itemize}
\end{frame}

\begin{frame}{How to analyse vowels?}
\begin{itemize}
\item[\checkmark] record sounds
\item[\checkmark] annotate sounds
\item[\checkmark] make an exploratory analysis
\item[\checkmark] extract duration and formant information from your data
\item create the plot
\end{itemize}
\end{frame}

\subsection{plotting}
\begin{frame}[fragile]{Plotting formant values with ggplot2}
\scriptsize
\begin{verbatim}
library(ggplot2)
setwd("...") # Put here path with the result.tsv file
df <- read.csv("result.txt", sep = "\t", fileEncoding = "UTF-8")
ggplot(data = df, aes(F2, F1, color = intervalname, label = intervalname))+
  geom_text(show.legend = F)+
  scale_y_reverse(position = "right")+
  scale_x_reverse(position = "top")
\end{verbatim}
\end{frame}

\begin{frame}[fragile]{Plotting formant values with ggplot2}
\begin{center}
\includegraphics[width=0.7\linewidth]{06-ggplot.png}
\end{center}
\end{frame}

\begin{frame}{How to analyse vowels?}
\begin{itemize}
\item[\checkmark] record sounds
\item[\checkmark] annotate sounds
\item[\checkmark] make an exploratory analysis
\item[\checkmark] extract duration and formant information from your data
\item[\checkmark] create the plot
\end{itemize}
\end{frame}

\section{R packages}
\subsection{vowels}
\begin{frame}{vowels}
\begin{itemize}
\item Version: 1.2-1
\item Date: 2014-11-14
\item Author: Tyler Kendall and Erik R. Thomas, \citep{kendall14}
\end{itemize}
\vfill
\texttt{install.packages("vowels")}
\end{frame}

\subsection{phonTools}
\begin{frame}{phonTools}
\begin{itemize}
\item Version: 0.2-2.1
\item Date: 2015-07-30
\item Author: Santiago Barreda, \citep{barreda15}
\end{itemize}
\vfill
\texttt{install.packages("phonTools")}
\end{frame}

\section{}
\begin{frame}
{\huge Thank you!\bigskip\\
\normalsize Please, don't hesitate to write me\\
agricolamz@gmail.com
\vspace{-130pt}}
\end{frame}
\begin{frame}{Reference}
\footnotesize
\bibliographystyle{chicago}
\bibliography{bibliography}
\end{frame}
\end{document}
